% test_document_structure.tex - Full document with sections and environments
\documentclass{article}
\usepackage{amsmath}
\usepackage{amssymb}
\usepackage{amsthm}

\newtheorem{theorem}{Theorem}
\newtheorem{lemma}[theorem]{Lemma}
\newtheorem{corollary}[theorem]{Corollary}
\newtheorem{definition}{Definition}

\title{Mathematical Analysis}
\author{Test Author}
\date{\today}

\begin{document}

\maketitle

\begin{abstract}
This document tests various LaTeX structural elements including theorems,
definitions, proofs, and mathematical environments.
\end{abstract}

\tableofcontents

\section{Introduction}

Let us consider the foundations of calculus.

\section{Limits and Continuity}

\begin{definition}
A function $f: \mathbb{R} \to \mathbb{R}$ is \emph{continuous} at $x = a$ if
$$\lim_{x \to a} f(x) = f(a)$$
\end{definition}

\begin{theorem}[Intermediate Value Theorem]
If $f$ is continuous on $[a,b]$ and $y$ is between $f(a)$ and $f(b)$,
then there exists $c \in (a,b)$ such that $f(c) = y$.
\end{theorem}

\begin{proof}
Consider the set $S = \{x \in [a,b] : f(x) \leq y\}$. Since $f(a) \leq y$,
we have $a \in S$, so $S \neq \emptyset$. Also $S$ is bounded above by $b$.

Let $c = \sup S$. We claim $f(c) = y$.

Suppose $f(c) < y$. By continuity, there exists $\delta > 0$ such that
$f(x) < y$ for all $x \in (c - \delta, c + \delta)$. This contradicts
$c = \sup S$.

Similarly, $f(c) > y$ leads to a contradiction. Therefore $f(c) = y$.
\end{proof}

\section{Differentiation}

\begin{definition}
The \emph{derivative} of $f$ at $x$ is
$$f'(x) = \lim_{h \to 0} \frac{f(x+h) - f(x)}{h}$$
when this limit exists.
\end{definition}

\begin{theorem}[Chain Rule]
If $g$ is differentiable at $x$ and $f$ is differentiable at $g(x)$, then
$$(f \circ g)'(x) = f'(g(x)) \cdot g'(x)$$
\end{theorem}

\begin{lemma}
If $f$ is differentiable at $x$, then $f$ is continuous at $x$.
\end{lemma}

\begin{corollary}
A function with a discontinuity at $x$ cannot be differentiable at $x$.
\end{corollary}

\section{Integration}

\begin{theorem}[Fundamental Theorem of Calculus]
Let $f$ be continuous on $[a,b]$ and let
$$F(x) = \int_a^x f(t) \, dt$$
Then:
\begin{enumerate}
\item $F$ is continuous on $[a,b]$
\item $F$ is differentiable on $(a,b)$ with $F'(x) = f(x)$
\item If $G$ is any antiderivative of $f$, then
$$\int_a^b f(x) \, dx = G(b) - G(a)$$
\end{enumerate}
\end{theorem}

\begin{proof}
(Part 2) Fix $x \in (a,b)$ and consider
\begin{align*}
\frac{F(x+h) - F(x)}{h} &= \frac{1}{h} \left( \int_a^{x+h} f(t) \, dt - \int_a^x f(t) \, dt \right) \\
&= \frac{1}{h} \int_x^{x+h} f(t) \, dt
\end{align*}
By the Mean Value Theorem for integrals, there exists $c_h$ between $x$ and $x+h$ such that
$$\frac{1}{h} \int_x^{x+h} f(t) \, dt = f(c_h)$$
As $h \to 0$, we have $c_h \to x$, so by continuity $f(c_h) \to f(x)$.
\end{proof}

\appendix

\section{Notation}

\begin{itemize}
\item $\mathbb{N}$ -- Natural numbers
\item $\mathbb{Z}$ -- Integers
\item $\mathbb{Q}$ -- Rational numbers
\item $\mathbb{R}$ -- Real numbers
\item $\mathbb{C}$ -- Complex numbers
\end{itemize}

\end{document}
