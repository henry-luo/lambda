% Comprehensive LaTeX test document
\documentclass[12pt,a4paper,twoside]{article}

% Package imports with options
\usepackage[utf8]{inputenc}
\usepackage[T1]{fontenc}
\usepackage{amsmath,amsfonts,amssymb}
\usepackage{graphicx}
\usepackage[colorlinks=true,linkcolor=blue]{hyperref}
\usepackage{geometry}
\usepackage{fancyhdr}

% Document metadata
\title{Comprehensive LaTeX Test Document}
\author{Test Author \and Second Author}
\date{\today}

% Custom commands
\newcommand{\mycommand}[1]{\textbf{#1}}
\newenvironment{myenv}{\begin{center}}{\end{center}}

\begin{document}

\maketitle
\tableofcontents
\newpage

\begin{abstract}
This comprehensive test document contains various LaTeX constructs including advanced formatting, mathematical expressions, tables, figures, cross-references, and special environments. The purpose is to test the robustness of the LaTeX parser and formatter implementation.
\end{abstract}

\section{Introduction}

This document tests various LaTeX features. We start with basic text formatting and progress to more complex constructs.

\subsection{Text Formatting}

Here we demonstrate various text formatting options:
\begin{itemize}
    \item \textbf{Bold text} and \textit{italic text}
    \item \texttt{Monospace text} and \underline{underlined text}
    \item \emph{Emphasized text} and \textsf{Sans serif text}
    \item \textsc{Small caps} and \textsl{Slanted text}
    \item \verb|Verbatim text| with special characters
\end{itemize}

\subsection{Font Sizes and Styles}

Different font sizes:
{\tiny Tiny text} {\scriptsize Script size} {\footnotesize Footnote size} {\small Small text} Normal size {\large Large text} {\Large Larger text} {\LARGE Even larger} {\huge Huge text} {\Huge Enormous text}

\section{Lists and Enumerations}

\subsection{Unordered Lists}

\begin{itemize}
    \item First level item
    \begin{itemize}
        \item Second level item
        \item Another second level item
        \begin{itemize}
            \item Third level item
            \item Another third level item
        \end{itemize}
    \end{itemize}
    \item Back to first level
\end{itemize}

\subsection{Ordered Lists}

\begin{enumerate}
    \item First numbered item
    \item Second numbered item
    \begin{enumerate}
        \item Nested numbered item
        \item Another nested item
    \end{enumerate}
    \item Third numbered item
\end{enumerate}

\subsection{Description Lists}

\begin{description}
    \item[Term 1] Definition of the first term
    \item[Term 2] Definition of the second term with more detailed explanation
    \item[Long Term Name] Definition with a longer term name
\end{description}

\section{Mathematical Expressions}

\subsection{Inline Mathematics}

Simple inline math: $x + y = z$, $\alpha + \beta = \gamma$, and $\sum_{i=1}^{n} x_i$.

More complex inline expressions: $\int_{-\infty}^{\infty} e^{-x^2} dx = \sqrt{\pi}$ and $\frac{\partial f}{\partial x} = \lim_{h \to 0} \frac{f(x+h) - f(x)}{h}$.

\subsection{Display Mathematics}

Simple display equation:
$$E = mc^2$$

Numbered equation:
\begin{equation}
\label{eq:quadratic}
ax^2 + bx + c = 0
\end{equation}

The solutions to equation \ref{eq:quadratic} are:
\begin{equation}
x = \frac{-b \pm \sqrt{b^2 - 4ac}}{2a}
\end{equation}

Multi-line equations:
\begin{align}
f(x) &= ax^2 + bx + c \\
f'(x) &= 2ax + b \\
f''(x) &= 2a
\end{align}

Matrix example:
$$
\begin{pmatrix}
a & b & c \\
d & e & f \\
g & h & i
\end{pmatrix}
\begin{pmatrix}
x \\
y \\
z
\end{pmatrix}
=
\begin{pmatrix}
ax + by + cz \\
dx + ey + fz \\
gx + hy + iz
\end{pmatrix}
$$

\section{Tables}

\subsection{Simple Table}

\begin{center}
\begin{tabular}{|l|c|r|}
\hline
Left & Center & Right \\
\hline
Item 1 & 100 & \$10.00 \\
Item 2 & 200 & \$20.00 \\
Item 3 & 300 & \$30.00 \\
\hline
\textbf{Total} & \textbf{600} & \textbf{\$60.00} \\
\hline
\end{tabular}
\end{center}

\subsection{Complex Table}

\begin{table}[htbp]
\centering
\caption{Sample data table with various formatting}
\label{tab:sample}
\begin{tabular}{|l|c|c|c|}
\hline
\textbf{Category} & \textbf{Value A} & \textbf{Value B} & \textbf{Total} \\
\hline
Category 1 & 10.5 & 20.3 & 30.8 \\
Category 2 & 15.2 & 25.7 & 40.9 \\
Category 3 & 12.8 & 18.4 & 31.2 \\
\hline
\textbf{Sum} & \textbf{38.5} & \textbf{64.4} & \textbf{102.9} \\
\hline
\end{tabular}
\end{table}

\section{Special Environments}

\subsection{Quotations}

Regular quote environment:
\begin{quote}
This is a quote environment that is used for shorter quotations. The text is indented from both margins.
\end{quote}

Quotation environment:
\begin{quotation}
This is a quotation environment used for longer quotations. Like the quote environment, it indents the text, but it also indents the first line of each paragraph.
\end{quotation}

\subsection{Verbatim Text}

\begin{verbatim}
This is verbatim text where
    whitespace is preserved
and special characters like \, {, }, $, %, #, &, _ work normally.
Code example:
    for (int i = 0; i < n; i++) {
        printf("Hello, World!\n");
    }
\end{verbatim}

\subsection{Center Environment}

\begin{center}
This text is centered.

Multiple lines can be centered
in the same environment.
\end{center}

\section{Cross-References and Citations}

See equation \ref{eq:quadratic} on page \pageref{eq:quadratic}.
Also refer to table \ref{tab:sample} for sample data.

\section{Special Characters and Symbols}

LaTeX special characters: \& \% \$ \# \_ \{ \} \textbackslash

Mathematical symbols: $\alpha, \beta, \gamma, \delta, \epsilon, \phi, \psi, \omega$

Arrows: $\leftarrow, \rightarrow, \leftrightarrow, \Leftarrow, \Rightarrow, \Leftrightarrow$

Set symbols: $\in, \notin, \subset, \supset, \cap, \cup, \emptyset, \infty$

\section{Advanced Features}

\subsection{Footnotes}

This text has a footnote\footnote{This is a footnote with additional information.} and another one\footnote{This is another footnote with more details about the topic.}.

\subsection{Custom Commands}

Using custom command: \mycommand{This text is processed by a custom command}.

\subsection{Line Breaks and Spacing}

Force line break here \\
and continue on next line.

Add some vertical space:

\vspace{1cm}

Text after vertical space.

Horizontal space: Text\hspace{2cm}with horizontal space.

\subsection{Comments}

This is visible text.
% This is a comment that should not appear in output
More visible text after comment.

\section{Conclusion}

This comprehensive test document covers most common LaTeX constructs and should thoroughly test the parser and formatter implementation. The document includes text formatting, mathematical expressions, tables, lists, special environments, cross-references, and various other LaTeX features.

\end{document}
