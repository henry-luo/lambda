\documentclass[11pt,letterpaper]{article}
\usepackage[utf8]{inputenc}
\usepackage{amsmath,amsfonts,amssymb}
\usepackage{geometry}
\usepackage{graphicx}
\usepackage{hyperref}

\geometry{margin=1in}

\title{Comprehensive LaTeX Document Test for Phase 3}
\author{Lambda Typesetting System Enhanced}
\date{\today}

\begin{document}

\maketitle

\tableofcontents

\section{Introduction}

This document serves as a comprehensive test for the enhanced LaTeX typesetting system in Phase 3. It includes advanced document features such as complex formatting, mathematical expressions, lists, tables, and various text styles.

\subsection{Document Structure Features}

The document demonstrates:
\begin{itemize}
    \item Multiple section levels with proper hierarchy
    \item Table of contents generation
    \item Advanced text formatting capabilities
    \item Mathematical expressions both inline and display
    \item Complex list structures with multiple nesting levels
    \item Tables with borders and formatting
    \item References and cross-references
\end{itemize}

\section{Text Formatting and Typography}

\subsection{Font Styles and Emphasis}

This section demonstrates various text formatting options available in LaTeX:

\textbf{Bold text} is used for strong emphasis, while \textit{italic text} provides subtle emphasis. We can combine them as \textbf{\textit{bold italic text}}. For code or technical terms, we use \texttt{monospace font}.

\subsubsection{Small Capitals and Other Styles}

LaTeX also supports \textsc{Small Capitals} and various other formatting options. We can create \underline{underlined text} when needed.

\paragraph{Paragraph Level Heading} This demonstrates a paragraph-level heading which is smaller than subsections.

\subparagraph{Subparagraph Level} This is the smallest standard sectioning level in LaTeX.

\subsection{Font Size Demonstrations}

Different font sizes can be achieved:

{\tiny This is tiny text}

{\scriptsize This is script size}

{\footnotesize This is footnote size}

{\small This is small text}

{\normalsize This is normal size}

{\large This is large text}

{\Large This is Large text}

{\LARGE This is LARGE text}

{\huge This is huge text}

{\Huge This is Huge text}

\section{Mathematical Expressions}

\subsection{Inline Mathematics}

Mathematics can be embedded inline within text, such as $E = mc^2$, or $\pi \approx 3.14159$, or even more complex expressions like $\sum_{i=1}^{n} x_i^2 = x_1^2 + x_2^2 + \cdots + x_n^2$.

The quadratic formula is given by $x = \frac{-b \pm \sqrt{b^2 - 4ac}}{2a}$ for a quadratic equation $ax^2 + bx + c = 0$.

\subsection{Display Mathematics}

For more prominent mathematical expressions, we use display mode:

\begin{equation}
\int_{-\infty}^{\infty} e^{-x^2} dx = \sqrt{\pi}
\end{equation}

\begin{equation}
\sum_{n=1}^{\infty} \frac{1}{n^2} = \frac{\pi^2}{6}
\end{equation}

\begin{align}
\nabla \times \mathbf{E} &= -\frac{\partial \mathbf{B}}{\partial t} \\
\nabla \times \mathbf{B} &= \mu_0\mathbf{J} + \mu_0\epsilon_0\frac{\partial \mathbf{E}}{\partial t} \\
\nabla \cdot \mathbf{E} &= \frac{\rho}{\epsilon_0} \\
\nabla \cdot \mathbf{B} &= 0
\end{align}

\subsection{Complex Mathematical Structures}

\begin{equation}
\mathbf{A} = \begin{pmatrix}
a_{11} & a_{12} & \cdots & a_{1n} \\
a_{21} & a_{22} & \cdots & a_{2n} \\
\vdots & \vdots & \ddots & \vdots \\
a_{m1} & a_{m2} & \cdots & a_{mn}
\end{pmatrix}
\end{equation}

\begin{equation}
f(x) = \begin{cases}
x^2 & \text{if } x \geq 0 \\
-x^2 & \text{if } x < 0
\end{cases}
\end{equation}

\section{Lists and Enumerations}

\subsection{Unordered Lists}

Basic itemize list:
\begin{itemize}
    \item First item
    \item Second item with longer text that may wrap to multiple lines to demonstrate text flow
    \item Third item
    \item Fourth item with nested items:
    \begin{itemize}
        \item Nested item one
        \item Nested item two with even more text to show wrapping behavior
        \item Nested item three with deeper nesting:
        \begin{itemize}
            \item Deep nested item one
            \item Deep nested item two
        \end{itemize}
    \end{itemize}
    \item Fifth item back at the top level
\end{itemize}

\subsection{Ordered Lists}

Basic enumerate list:
\begin{enumerate}
    \item First numbered item
    \item Second numbered item
    \item Third numbered item with nested numbering:
    \begin{enumerate}
        \item Nested numbered item A
        \item Nested numbered item B
        \item Nested numbered item C with deeper nesting:
        \begin{enumerate}
            \item Deep nested numbered item i
            \item Deep nested numbered item ii
        \end{enumerate}
    \end{enumerate}
    \item Fourth numbered item
\end{enumerate}

\subsection{Description Lists}

\begin{description}
    \item[Term One] Definition of term one with detailed explanation
    \item[Term Two] Definition of term two
    \item[Complex Term] This term has a longer definition that spans multiple lines and demonstrates how description lists handle longer text content
    \item[Mathematical Term] $f(x) = x^2 + 2x + 1$ represents a quadratic function
\end{description}

\section{Tables and Tabular Data}

\subsection{Simple Tables}

\begin{table}[h]
\centering
\begin{tabular}{|l|c|r|}
\hline
Left Aligned & Center Aligned & Right Aligned \\
\hline
Data 1 & Data 2 & Data 3 \\
Data 4 & Data 5 & Data 6 \\
Data 7 & Data 8 & Data 9 \\
\hline
\end{tabular}
\caption{Simple table with borders and different alignments}
\label{tab:simple}
\end{table}

\subsection{Complex Tables}

\begin{table}[h]
\centering
\begin{tabular}{|l|c|c|c|c|}
\hline
\textbf{Function} & \textbf{Domain} & \textbf{Range} & \textbf{Derivative} & \textbf{Integral} \\
\hline
$\sin(x)$ & $\mathbb{R}$ & $[-1,1]$ & $\cos(x)$ & $-\cos(x) + C$ \\
\hline
$\cos(x)$ & $\mathbb{R}$ & $[-1,1]$ & $-\sin(x)$ & $\sin(x) + C$ \\
\hline
$e^x$ & $\mathbb{R}$ & $(0,\infty)$ & $e^x$ & $e^x + C$ \\
\hline
$\ln(x)$ & $(0,\infty)$ & $\mathbb{R}$ & $\frac{1}{x}$ & $x\ln(x) - x + C$ \\
\hline
\end{tabular}
\caption{Mathematical functions and their properties}
\label{tab:functions}
\end{table}

\section{Advanced Features}

\subsection{Quotations and Verse}

Here is a block quote:

\begin{quote}
This is a quoted text that demonstrates block quotation formatting. It should be indented from both sides and possibly formatted differently from regular paragraph text.
\end{quote}

For longer quotations:

\begin{quotation}
This is a longer quotation that may span multiple paragraphs. The quotation environment is designed for longer quoted material.

This is the second paragraph of the longer quotation, demonstrating how multi-paragraph quotes should be formatted.
\end{quotation}

\subsection{Code and Verbatim Text}

Inline code can be written as \verb|printf("Hello World")|.

For longer code blocks:

\begin{verbatim}
#include <stdio.h>

int main() {
    printf("Hello, World!\n");
    return 0;
}
\end{verbatim}

\subsection{Footnotes and References}

This document demonstrates footnotes\footnote{This is a footnote that provides additional information.} and references to tables like Table~\ref{tab:simple} and Table~\ref{tab:functions}.

We can also reference sections, such as Section~\ref{sec:conclusion} which appears later in the document.

\section{Special Characters and Symbols}

LaTeX handles various special characters and symbols:

Mathematical symbols: $\alpha, \beta, \gamma, \Delta, \Sigma, \infty, \partial, \nabla$

Greek letters: $\Alpha, \Beta, \Gamma, \Delta, \Epsilon, \Zeta, \Eta, \Theta$

Special symbols: \&, \%, \$, \#, \textbackslash, \textasciicircum, \textasciitilde

Accented characters: caf\'{e}, na\"{i}ve, r\^{o}le, \c{c}a va

\section{Page Layout and Spacing}

This section tests various spacing and layout features:

\vspace{1em}

This paragraph has extra vertical space above it.

\hspace{2em} This line starts with horizontal space.

\noindent This paragraph has no indentation at the beginning.

\begin{center}
This text is centered on the page.
\end{center}

\begin{flushleft}
This text is flush left aligned.
\end{flushleft}

\begin{flushright}
This text is flush right aligned.
\end{flushright}

\section{Conclusion}
\label{sec:conclusion}

This comprehensive test document demonstrates the enhanced capabilities of the Lambda LaTeX typesetting system in Phase 3. The document includes:

\begin{itemize}
    \item Complex document structure with multiple section levels
    \item Advanced text formatting and typography
    \item Sophisticated mathematical expressions
    \item Multi-level lists and enumerations
    \item Tables with various formatting options
    \item Special characters and symbols
    \item Page layout and spacing features
    \item Cross-references and footnotes
\end{itemize}

The successful rendering of this document in PDF format validates the enhanced typesetting capabilities and demonstrates readiness for production use.

\end{document}
