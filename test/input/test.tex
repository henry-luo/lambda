% Test LaTeX document for parser testing
\documentclass[12pt,a4paper]{article}

% Package imports
\usepackage[utf8]{inputenc}
\usepackage{amsmath}
\usepackage{amsfonts}
\usepackage{amssymb}
\usepackage{graphicx}
\usepackage{hyperref}

% Document metadata
\title{LaTeX Parser Test Document}
\author{Test Author}
\date{\today}

\begin{document}

\maketitle

\begin{abstract}
This is a comprehensive test document for the LaTeX parser implementation. It includes various LaTeX constructs such as document structure, text formatting, mathematical expressions, lists, tables, and special characters.
\end{abstract}

\tableofcontents

\section{Introduction}

This document serves as a test case for the LaTeX parser. It demonstrates various LaTeX features and constructs that the parser should be able to handle correctly.

\subsection{Basic Text Formatting}

Here we test basic text formatting commands:
\begin{itemize}
    \item \textbf{Bold text} using textbf command
    \item \textit{Italic text} using textit command
    \item \texttt{Monospace text} using texttt command
    \item \underline{Underlined text} using underline command
    \item \emph{Emphasized text} using emph command
\end{itemize}

\subsection{Font Sizes}

Different font sizes can be used:
{\tiny This is tiny text.}
{\scriptsize This is script size.}
{\footnotesize This is footnote size.}
{\small This is small text.}
Normal size text.
{\large This is large text.}
{\Large This is Large text.}
{\LARGE This is LARGE text.}
{\huge This is huge text.}
{\Huge This is Huge text.}

\section{Mathematical Expressions}

\subsection{Inline Mathematics}

Here are some inline mathematical expressions: $x + y = z$, $\alpha + \beta = \gamma$, and $\int_0^1 f(x) dx$.

The quadratic formula is: $x = \frac{-b \pm \sqrt{b^2 - 4ac}}{2a}$

\subsection{Display Mathematics}

Display mathematics can be written using double dollar signs:

$$E = mc^2$$

$$\sum_{i=1}^{n} i = \frac{n(n+1)}{2}$$

\subsection{Equation Environment}

\begin{equation}
\int_{-\infty}^{\infty} e^{-x^2} dx = \sqrt{\pi}
\end{equation}

\begin{align}
x &= a + b \\
y &= c + d \\
z &= e + f
\end{align}

\section{Lists}

\subsection{Unordered Lists}

\begin{itemize}
    \item First item
    \item Second item
    \begin{itemize}
        \item Nested item 1
        \item Nested item 2
    \end{itemize}
    \item Third item
\end{itemize}

\subsection{Ordered Lists}

\begin{enumerate}
    \item First numbered item
    \item Second numbered item
    \begin{enumerate}
        \item Nested numbered item 1
        \item Nested numbered item 2
    \end{enumerate}
    \item Third numbered item
\end{enumerate}

\subsection{Description Lists}

\begin{description}
    \item[Term 1] Definition of the first term
    \item[Term 2] Definition of the second term
    \item[Long Term Name] Definition of a term with a longer name
\end{description}

\section{Special Characters and Symbols}

LaTeX has many special characters that need proper handling:

\begin{itemize}
    \item Escaped characters: \{ \} \$ \& \# \^ \_ \% \~
    \item Mathematical symbols: $\alpha$, $\beta$, $\gamma$, $\delta$, $\pi$, $\sigma$, $\infty$
    \item Operators: $\cdot$, $\times$, $\div$, $\pm$, $\mp$
    \item Relations: $\leq$, $\geq$, $\neq$, $\approx$, $\equiv$
    \item Arrows: $\leftarrow$, $\rightarrow$, $\leftrightarrow$, $\Leftarrow$, $\Rightarrow$
\end{itemize}

\section{Environments}

\subsection{Quote Environment}

\begin{quote}
This is a quoted text block that demonstrates the quote environment. It is typically used for longer quotations that should be set apart from the main text.
\end{quote}

\subsection{Verbatim Environment}

\begin{verbatim}
This is verbatim text where all
    special characters like \textbf{bold}
    and $math$ are treated literally.
\end{verbatim}

\subsection{Center Environment}

\begin{center}
This text is centered using the center environment.
\end{center}

\section{Cross-References}

We can reference sections like Section~\ref{sec:introduction} or equations like Equation~\ref{eq:quadratic}.

\begin{equation}
\label{eq:quadratic}
ax^2 + bx + c = 0
\end{equation}

\section{Comments and Spacing}

% This is a comment that should be ignored by the parser
This line comes after a comment.

LaTeX handles spacing automatically, but we can force spacing with commands like \quad and \qquad.

Here is some text\quad with a quad space\qquad and a qquad space.

\section{Complex Nested Structures}

\begin{itemize}
    \item Outer item with \textbf{bold text}
    \item Item with math: $\sum_{i=1}^{n} x_i$
    \item Item with nested environment:
    \begin{quote}
        This is a quote within a list item, demonstrating nested structures.
    \end{quote}
    \item Final outer item
\end{itemize}

\section{Document Commands}

Some common document-level commands:
\begin{itemize}
    \item \LaTeX{} - The LaTeX logo
    \item \TeX{} - The TeX logo  
    \item \ldots{} - Ellipsis dots
    \item Footnote\footnote{This is a footnote example}
\end{itemize}

\section{Conclusion}

This document provides a comprehensive test case for LaTeX parsing, covering:
\begin{enumerate}
    \item Document structure (sections, subsections)
    \item Text formatting (bold, italic, fonts)
    \item Mathematical expressions (inline and display)
    \item Lists (itemize, enumerate, description)
    \item Special characters and symbols
    \item Various environments
    \item Comments and spacing
    \item Cross-references
    \item Nested structures
\end{enumerate}

The parser should be able to handle all these constructs and represent them appropriately in the parsed structure.

\end{document}
